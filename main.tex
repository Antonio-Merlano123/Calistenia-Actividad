
\documentclass{article}
\usepackage[utf8]{inputenc}
\usepackage[spanish]{babel}
\usepackage{listings}
\usepackage{graphicx}
\graphicspath{ {images/} }


\begin{document}

\begin{titlepage}
    \begin{center}
        \vspace*{1cm}
            
        \Huge
        \textbf{Trabajo Informatica II}
            
        \vspace{0.5cm}
        \LARGE
        Calistenia
            
        \vspace{1.5cm}
            
        \textbf{Antonio Carlos Merlano Ricardo}
            
        \vfill
            
        \vspace{0.8cm}
            
        \Large
        Despartamento de Ingeniería Electrónica y Telecomunicaciones\\
        Universidad de Antioquia\\
        Medellín\\
        Marzo de 2021
            
    \end{center}
\end{titlepage}

\tableofcontents
\newpage
\section{Pasos  para llevar obejetos de posición A a una posición B}


\section{Pasos.} \\

Recordamos que necesitamos una hoja lisa para  usarlo arriba de la superficie que vayamos a usar.\\
\\

1.	Tomar con la mano más hábil las dos tarjetas.\\
\\

2.	Recordamos que si estamos en una superficie no tan firme como una mesa tratar de no afirmarse de ella.\\
\\

3.	Agarrar  los lados largos de cada uno de las tarjetas con el dedo pulgar y dedo medio respectivamente, y situarlo en una posición vertical.\\
\\

4.	Apoyamos el lado descubierto de la tarjeta con el dedo indice.\\
\\

5.	 Desplazar con dedo anular  mientras movemos sin tocar totalmente el papel y con ayuda del dedo pulgar para obtener las tarjetas en una posición firme. 



\bibliographystyle{IEEEtran}


\end{document}
